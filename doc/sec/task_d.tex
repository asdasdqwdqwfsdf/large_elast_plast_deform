\section{Task D}
\label{sec:task-d}

The logarithmic strain model from chapter 5 \cite{Ekh2016} was implemented
in Matlab function \texttt{neo\_hooke\_plast\_log\_strain.m} (see 
section \ref{app:matlab-code}) and applied to solve the same boundary 
value problem as in section \ref{sec:task-c}.
Figure \ref{fig:forming-bvp-log-strain} compares the total reaction force
plotted against displacement of the top nodes for the two models.
The two models yield virtually the same graphs.
\begin{figure}[th]
  \centering
  % Reaction force
    \begin{tikzpicture}
      \begin{axis}[
        width = 0.95\textwidth,
        height=\axisdefaultheight,
        tick label style={/pgf/number format/fixed},
        try min ticks=6,
        minor tick num=1,
        grid=both,
        legend style={at={(0.5,0)},
          anchor=south},
        xlabel = {\(u_{x}\), [mm]},
        ylabel = {Reaction force, [N]},
        xmin=0, xmax=5,
        ]
        \addplot[mark=none, blue] table[skip first n=1]
        {data/force_displacement_forming.dat};
        \addlegendentry{Neo-Hooke}
        \addplot[mark=none, red, dashed] table[skip first n=1]
        {data/force_displacement_log_strain.dat};
        \addlegendentry{Neo-Hooke, log strain}
      \end{axis}
    \end{tikzpicture}
    \caption{Reaction force versus displacement.}
  \label{fig:forming-bvp-log-strain}
\end{figure}


%%% Local Variables:
%%% mode: latex
%%% TeX-master: "../main"
%%% End:
